%%%%%%%%%%%%%%%%%%%%%%%%%%%%%%%%%%%%%%%%%%%%%%%%%%%%%
%												    %
%	REDES DE COMPUTADORES						 %
%												    %
%	Novembro 2015								    %
%												    %
%	Angela Cardodo e Bruno Madeira					%
%   											    %	
%%%%%%%%%%%%%%%%%%%%%%%%%%%%%%%%%%%%%%%%%%%%%%%%%%%%%

\documentclass[11pt,a4paper,reqno]{report}
\linespread{1.2}


\usepackage{rotating}
\usepackage{tikz}
\usepackage[active]{srcltx}    
\usepackage{graphicx}
\usepackage{amsthm,amsfonts,amsmath,amssymb,indentfirst,mathrsfs,amscd}
\usepackage[mathscr]{eucal}
\usepackage{tensor}
%\usepackage[utf8x]{inputenc}

\usepackage[utf8]{inputenc}

\usepackage[portuges]{babel}
\usepackage[T1]{fontenc}
\usepackage{enumitem}
\setlist{nolistsep}
\usepackage{comment} 
\usepackage{tikz}
\usepackage[numbers,square, comma, sort&compress]{natbib}
\usepackage[nottoc,numbib]{tocbibind}
%\numberwithin{figure}{section}
\numberwithin{equation}{section}
\usepackage{scalefnt}
\usepackage[top=1cm, bottom=2cm, left=2cm, right=2cm]{geometry}
%\usepackage{tweaklist}
%\renewcommand{\itemhook}{\setlength{\topsep}{0pt}%
%	\setlength{\itemsep}{0pt}}
%\renewcommand{\enumhook}{\setlength{\topsep}{0pt}%
%	\setlength{\itemsep}{0pt}}
%\usepackage[colorlinks]{hyperref}
\usepackage{MnSymbol}
%\usepackage[pdfpagelabels,pagebackref,hypertexnames=true,plainpages=false,naturalnames]{hyperref}
\usepackage[naturalnames]{hyperref}
\usepackage{enumitem}
\usepackage{titling}
\newcommand{\subtitle}[1]{%
	\posttitle{%
	\par\end{center}
	\begin{center}\large#1\end{center}
	\vskip0.5em}%
}
\newcommand{\HRule}{\rule{\linewidth}{0.5mm}}
\usepackage{caption}
\usepackage{etoolbox}% http://ctan.org/pkg/etoolbox
\usepackage{complexity}

\usepackage[official]{eurosym}

\def\Cpp{C\raisebox{0.5ex}{\tiny\textbf{++}}}

\makeatletter
\def\@makechapterhead#1{%
  \vspace*{2\p@}%
  {\parindent \z@ \raggedright \normalfont
    \ifnum \c@secnumdepth >\m@ne
%        \huge\bfseries \@chapapp\space \thechapter
        \par\nobreak
        \vskip 5\p@
    \fi
    \interlinepenalty\@M
    \Huge \bfseries \thechapter\space #1\par\nobreak
    \vskip 20\p@
  }}
\def\@makeschapterhead#1{%
  \vspace*{2\p@}%
  {\parindent \z@ \raggedright
    \normalfont
    \interlinepenalty\@M
    \Huge \bfseries  #1\par\nobreak
%    \vskip 5\p@
  }}
\makeatother

\usepackage[toc,page]{appendix}

\addto\captionsportuges{%
  \renewcommand\appendixname{Anexo}
  \renewcommand\appendixpagename{Anexos}
}

\addto\captionsportuges{%
  \renewcommand\abstractname{\huge Sumário}  
}

\usepackage{verbatim}
\usepackage{color}
\definecolor{darkgray}{rgb}{0.41, 0.41, 0.41}
\definecolor{green}{rgb}{0.0, 0.5, 0.0}
\usepackage{listings}
\lstset{language=C++, 
    basicstyle=\linespread{0.8}\ttfamily,
    keywordstyle=\color{blue}\ttfamily,
	showstringspaces=false,
    stringstyle=\color{red}\ttfamily,
    commentstyle=\color{green}\ttfamily,
	identifierstyle=\color{darkgray}\ttfamily,
    morecomment=[l][\color{magenta}]{\#},
	tabsize=4,
    breaklines=true
}

\begin{document}



\begin{titlepage}
\begin{center}
 
\vspace*{3cm}

{\Large Redes de Computadores}\\[2cm]

% Title
{\Huge \bfseries Redes de Computadores  \\[1cm]}

% Author
{\large \^Angela Cardoso e Bruno Madeira}\\[2cm]

\includegraphics[width=10cm]{feup_logo.jpg}\\[2cm]


% Bottom of the page
{\large \today}

\end{center}
\end{titlepage}


%%%%%%%%%%%
% SUMARIO %
%%%%%%%%%%%
\begin{abstract}
	
Este relatório tem como objectivo reportar o segundo trabalho prático relativo a Redes de Computadores da Licenciatura com Mestrado em Engenharia Informátia e Computação que consiste na configuração de uma rede e na implementação de uma aplicação de download de ficheiros.


\end{abstract}

\tableofcontents

%%%%%%%%%%%%%%
% INTRODUCAO %
%%%%%%%%%%%%%%
\chapter{Introdução}

No âmbito da unidade curricular de Redes de Computadores foi-nos proposta a realização de um trabalho prático cujo objetivo principal era configurar uma rede e compreender os vários aspetos dessa configuração. Além de realizarmos essa configuração, implementamos também uma aplicação de download de ficheiros, por forma a testar uma parte da rede.

A primeira secção deste relatório incide sobre a aplicação desenvolvida. A aplicação de download de ficheiro foi implementada fora das aulas práticas e o relatório tenta esclarecer detalhes de implementação da mesma a fim de eliminar possíveis dúvidas que possam surgir face à mesma.

A segunda secção do relatório incide sob os sete exercícios realizados nas aulas práticas relacionados com a configuração de rede. O relatório evita relatar os exercícios detalhadamente uma vez que estes podem ser consultados no guião do trabalho e tenta focar-se mais na análise e interpretação dos resultados obtidos com o software \href{https://www.wireshark.org/}{Wireshark}.

Na análise de dados realizada na segunda secção do relatório pode ser útil consultar o anexo ~\ref{mac_addresses} que apresenta os endereços MAC dos tuxs. É importante referir também que apesar de deste relatório referir muitas vezes o tux2, da experiência 4 até à 7, qualquer referência ao tux2 corresponde na realidade ao tux3 uma vez que o tux2 deixou de estar disponivel apartir de dada altura. Para que o relatório respeite os nomes referidos no guião e usados nos anexos, mantendo a continuidade entre experiências, decidimos continuar a referir-nos ao terceiro computador usado na rede como sendo o tux2.  
	
No final do relatório apresentamos uma conclusão com as nossas considerações face ao mesmo e ao nossa prestação.
%%%%%%%%%%%%%%%
% APLICAÇÂO %
%%%%%%%%%%%%%%%
\chapter{Aplicação}

A aplicação desenvolvida realiza o download de um ficheiro fazendo uso do protocolo FTP, cuja especificação se encontra em  \href{https://www.ietf.org/rfc/rfc959.txt}{RFC959}.
Para tal são usadas duas sockets, uma para comandos e outra para dados, de acordo com o modelo descrito na secção 2.3 do RFC959.
Os comandos usados podem ser verificados na secção 4 (páginas 25 a 34)  e na página 47 do RFC959. É usado o comando PASV sendo que o servidor não usa a porta default para os dados (porta 20) e fica à espera que o cliente estabeleça a ligação.

Todas as funcionalidades desenvolvidas ligadas ao protocolo FTP podem ser verificadas no ficheiro \verb|ftp.c| e \verb|ftp.h| disponíveis nos anexos~\ref{FTPC} e~\ref{FTPH}. Apesar de existir uma função denominada \verb|ftp_abort| esta não envia um comando \verb|ABORT| (embora esta tenha sido a funcionalidade inicialmente pensada para o mesmo). Esta função apenas fecha as sockets em caso de erro.

Para efectuar ligação ao servidor a aplicação deve também receber um URL no formato descrito no \href{https://www.ietf.org/rfc/rfc1738.txt}{RFC1738}. Não consideramos utilizadores anónimos como é referido na secção 3.2.1. do RFC1738. No ficheiro \verb|downloader.c| (ver anexo~\ref{DOWNLOADERC}) é realizado o parsing do \verb|url| ficando guardado numa estrutura o nome de utilizador, a sua password, o nome do host, o caminho até ao ficheiro e o nome do ficheiro.

Uma vez realizado o parsing tenta-se obter o IP do destino e cria-se uma ligação TCP para a porta 21 do servidor a fim de enviar os comandos para pedir a recepção do ficheiro. As funções usadas para obter o IP e para estabelecer são as disponibilizadas nos exemplos do moodle da disciplina. A conexão é realizada com a função \verb|connect| e não com a função \verb|bind| uma vez que a aplicação está do lado do cliente. É utilizada a função \verb|gethostbyname| para obter o IP, que funciona mas está depreciada segundo o \href{http://beej.us/guide/bgnet/output/html/multipage/index.html}{Beej's Guide to Network Programming}.

Em termos de estrutura foram desenvolvidos apenas 4 múdulos que apresentamos seguidamente.
\begin{itemize} 
\item \verb|downloader| - Onde se encontra a função \verb|main| da aplicação. Também é responsável pelo parsing e por obter o IP do destino.
\item \verb|ftp| -  Implementa e disponibiliza comandos do protocolo FTP. Os file descriptors das sockets também se encontram neste módulo.
\item \verb|socket| - Apenas disponibiliza uma função para conectar sockets.
\item \verb|utilities| - Apenas disponibiliza funções auxiliares para debug.
\end{itemize}

%%%%%%%%%%%%%%%%%%%%
% LABS REALIZADOS%
%%%%%%%%%%%%%%%%%%%%
\chapter{Experiências}

\section{Experiência 1 - Configurar uma Rede IP}

Nesta experiência criou-se uma rede LAN com o tux1 e o tux4 na mesma rede, tendo sido configurados os seus endereços IP. Usando o comando \verb|ping| na etapa 7, pudemos verificar o envio de um comando ARP em broadcast pelo tux1 que procurava o endereço físico do tux4, necessário ao protocolo ethernet usado para poder comunicar dentro de uma mesma rede local. Seguidamente verificou-se a resposta do tux4 e foi realizado o ping com sucesso.

Atentando nos pacotes capturados com o Wireshark do anexo~\ref{ex1_headers}, é possível verificar que os pacotes ARP são identificáveis pelo cabeçalho Ethernet x0806, enquanto que os pacotes IP têm o cabeçalho x0800. As mensagens de ping podem ser identificadas pelo cabeçalho Ethernet correspondente ao protocolo IP e pelo cabeçalho de IP x01 que corresponde ao protocolo ICMP.

...\\ TODO frame length \\

Na lista de pacotes recebidos existe também pacotes do tipo loopback. Este são pacotes são redireccionados para a máquina que os emitiu, tipicamente com a finalidade de verificar se esta se encontra em estado operacional. Neste caso, os pacotes recebidos aparentam ser do switch, tendo como endereço de origem e destino o CiscoInc\_3a:f1:03.

\section{Experiência 2 - Implementar 2 LANs num switch}

Foram criadas duas LANs. A primeira, com o tux1 e o tux4 na rede 172.16.60.0 (máscara de 24 bits), corresponde à experiência 1. A outra, com o tux2, na rede 172.16.61.0. Foram atribuídos endereços IP às máquinas relativos à rede em que se deviam encontrar e configurando o switch de modo a funcionarem como 2 redes distintas. Constatou-se que apenas os computadores que se encontravam na mesma rede virtual local podiam comunicar entre si. 

Nos anexos~\ref{ex2_tux1ping_tux2} e~\ref{ex2_tux1ping_tux4} verifica-se que pings realizados do tux1 em broadcast (alínea 7 do trabalho prático) chegam ao tux4 mas não ao tux2. Similarmente, não foi possível encontrar pacotes de ICMP no tux1 e no tux4 quando realizado ping a partir do tux2 como se pode observar nos restantes anexos da Secção~\ref{ex_2}.

\section{Experiência 3 - Configurar um Router em Linux}

No seguimento da experiência anterior, foi configurada a rede de modo a que o tux4 funcionasse como um router entre as duas LANs criadas. O tux4.eth0 continuou com o endereço 172.16.60.254 e ao tux4.eth1 foi atribuído o endereço 172.16.61.253. Foram também reconfigurados o tux1 e tux2 de modo a fazerem uso do router (tux4) para poderem comunicar entre si.

Nas tabelas de encaminhamento (forwarding tables) do tux1 e tux2 aparecem, respectivamente, os gateways 172.16.60.254 e 172.16.61.253 para aceder à rede vizinha. Estes gateways são os endereços para os quais devem ser encaminhados os pacotes IP que apresentam um endereço da rede vizinha como destino. Os ARPs enviados quando o tux1 pretende comunicar com o tux2 (ou vise-versa), percorrem apenas a LAN na qual foram emitidos com o objectivo de descobrir o endereço MAC do gateway. Os pacotes capturados pelo Wireshark no anexo~\ref{ex3_eth0} ilustram esta situação. No anexo referido estamos à escuta no tux4.eth0 e podemos verificar que é recebido um ARP de origem no tux1 a perguntar pelo endereço MAC do tux4.eth0. O tux1 quer realizar ping ao tux2 como se pode concluir pelos pacotes ICMP seguintes, e, só o faz depois de receber a resposta do tux4 ao seu ARP, que é necessário ao protocolo ethernet na camada de enlace (data-link layer).

Relativamente aos endereços dos pacotes ICMP é possível verificar que apresentam sempre o mesmo endereço IP de origem e destino na camada de rede (network layer), mas que o endereço MAC de origem e destino varia consoante a rede em que se encontram. Um pacote de ping ICMP proveniente do tux1 para o o tux2 apresenta inicialmente o endereço de origem do tux1.eth0 e de destino o tux4.eth0 (gateway). Depois de recebido pelo gateway (tux4) é enviado para o tux2 com os endereços MAC de origem em tux4.eth1 e destino tux2.eth0. O anexo~\ref{ex3_eth1} mostra esta última situação.



\section{Experiência 4 - Configurar um Router Comercial e Implementar NAT}
...TODO...\\
O NAT (Network Address Translation) permite criar uma separação entre uma rede LAN e uma outra rede (tipicamente maior, WAN por exemplo). Esta separação permite usar IPs dentro da LAN que podem já estar em uso fora desta. Funciona como solução ao limite de endereços do IPv4 e confere alguma segurança adicional à rede não permitindo acessos directos às máquinas desta. Na prática ele mapeia portas do gateway a pares de endereço e porta dentro da LAN. Na experiência 7 veremos um pouco melhor tudo isto.\\
...blababla...\\
~\ref{ex4_redirect}.
...blablabla...\\
\section{Experiência 5 - DNS}

\section{Experiência 6 - Conexões TCP}

Nesta experiência usámos a aplicação desenvolvida para realizar o download de um ficheiro. Foi chamada a aplicação inicialmente no tux1 e seguidamente após um pequeno intervalo de tempo no tux2. Sendo assim, o tux1 começou sozinho, depois há uma parte em que ambos os downloads são simultâneos e, finalmente, o tux2 termina o download sozinho.

Como esperado, devido ao protocolo FTP, pudemos verificar o \emph{3-way handshake} de duas conexões TCP. O primeiro relativo à ligação usada para envio de comandos e o segundo relativo a de envio de dados que podem ser verificados no anexo~\ref{ex6_tux1_handshakes}. O estabelecimento de conexão consiste no pedido do cliente ao servidor para estabelecer ligação (SYN) seguido da resposta do servidor (SYN, ACK) e de uma confirmação final pelo cliente (ACK) que podem ser melhor observados no anexo~\ref{ex6_tux1_1sthandshake}, onde é mostrado também o número de sequência e de confirmação em cada pacote. \\

O mecanismo de ARQ (Automatic Repeat Request) do protocolo TCP é uma variante do Go-Back-N onde o servidor envia confirmações relativas a cada segmento que recebe.

TODO...\\

Detalhes adicionais relativos ao protocolo TCP podem ser consultado no \href{https://www.ietf.org/rfc/rfc793.txt}{RFC793}.\\

Pouco depois do tux1 atingir o seu plateu máximo de tráfego, entre os 14 e 15 segundos do gráfico~\ref{ex6_a5_1io}, podem ser observadas vários pacotes do tipo [duplicate ACK], [Previous Segment not captured], [Fast Retransmission] e [Retransmission] que parecem indicar congestionamento.
Segundo o \href{http://www.ietf.org/rfc/rfc2581.txt}{RFC2581} o receptor deve enviar um duplicate ACK quando é recebido um segmento fora de ordem e pode ocorrer uma retransmissão, \emph{fast retransmit}, após a recepção de 3 confirmações duplicadas (duplicate ACKs) pelo transmissor. Na experiência foram capturados pelo Wireshark pacotes que parecem demonstrar este comportamento como se pode pelo anexo~\ref{ex6_retrans}.

TODO...\\

Na realização da última alínea pudemos verificar que a recepção de dados quando usada uma segunda ligação no tux2 era afectada. Pode observar-se nos anexos~\ref{ex6_a5_1io} e~\ref{ex6_a5_2io} que a recepção  tende para um plateu máximo no tux1 que é quebrado devido à ligação estabelecida pelo tux2. Observando o gráfico relativo ao tux2 podemos ver que este atinge um plateu máximo perto do final da sua ligação, que ocorre devido ao tux1 já ter terminado o download. Além deste plateu máximo podemos verificar que os gráficos são complementares no sentido em que a soma das funções dos dois gráficos, alinhando-os consoante os seus pontos mínimos e máximos dado que as leituras em Wireshark não foram iniciadas em simultâneo, resulta aproximadamente numa função constante que apresenta uma recepção entre 10000 e 12000 packets por segundo.

\section{Experiência 7 - Implementar NAT em Linux}

Experiência implementamos NAT no tux4 e geramos diferentes tipos de tráfego para interne. Foram usados os comando wget, traceroute e ping sendo consecutivamente observado o tráfego no tux4.eth0 e no tux4.eth1.\\

Verificou-se que usando o NAT no tux4 que os endereços IP de origem e destino da camada de Rede nos pacotes TCP variavam consoante a rede em que se encontravam, tal como era esperado ao usar NAT. O encaminhamento só é realizado devido às portas indicadas na camada de transporte (TCP) sendo que o tux4 re-encaminhou para o tux1 pacotes associados à porta 37351 como se pode ver no anexo~\ref{ex7_tcp}.\\

Foram enviados pacotes UDP ao realizar o traceroute. O protocolo UDP não é orientado a ligações e não é fiável ao contrário do TCP. No protocolo UDP não existem confirmações de pacotes nem retransmissões ou outros mecanismos que garantem a entrega de dados ao destinatário. Esta propriedade pode ser observedada no anexo ~\ref{ex7_udp} onde não foi recebida resposta a alguns dos pacotes UDP enviados. O \href{https://www.ietf.org/rfc/rfc768.txt}{RFC768} apresenta detalhes adicionais relativos ao protocolo UDP.\\

Foi observada a recepção de pacotes ICMP como resposta aos pacotes de traceroute e de ping. O protocolo ICMP não faz uso de portas como o UDP e TCP sendo que usando NAT só é possível realizar o encaminhamento correctamente devido ao uso de um "identifier" como referido na página 15 do \href{https://www.ietf.org/rfc/rfc792.txt}{RFC792}. O anexo ~\ref{ex7_ping} mostra um par de pacotes ping onde se pode observar que apresentam o mesmo "identifier". Mais detalhes relativos ao uso de ICMP com NAT estão disponíveis na secção 3 do \href{https://www.ietf.org/rfc/rfc5508.txt}{RFC5508} . 

%%%%%%%%%%%%%%
% CONCLUSOES %
%%%%%%%%%%%%%%
\chapter{Conclusões}

%%%%%%%%%%%%%%
% NOTAS IMPORTANTES %
%%%%%%%%%%%%%%
%\chapter{Esclarecimentos}

%Apesar de deste relatório referir muitas vezes o tux2, da experiência 4 até à 7, qualquer referência ao tux2 corresponde na realidade ao tux3 uma vez que o tux2 deixou de estar disponivel apartir de dada altura. Para que o relatório respeite os nomes referidos no guião e usados nos anexos, mantendo a continuidade entre experiências, decidimos continuar a referir-nos ao terceiro computador usado na rede como sendo o tux2.  

%%%%%%%%%%%%%%%%
% BIBLIOGRAPHY %
%%%%%%%%%%%%%%%%


\begin{appendices}

%%%%%%%%%%%%%%%%%%%%%%%%%%%
% APENDICE - VARIAS %
%%%%%%%%%%%%%%%%%%%%%%%%%%%
\chapter{Enderaços MAC}
\label{mac_addresses}
\begin{itemize} 
\item tux1 eth0: 00:0f:fe:8c:af:71
\item tux2 eth0: 00:21:5a:5a:7d:9c
\item tux3 eth0: 00:21:5a:61:2f:4e
\item tux4 eth0: 00:21:5a:c5:61:bb
\item tux4 eth1: 00:c0:df:04:20:8c
\end{itemize}

\chapter{Console logs}%XXXXXXXXXXXXXXXXXXXXXXXXXXXXXXXXXXXX

\section{Ex4 alínea 4 (redirect)}
\label{ex4_redirect}
\begin{lstlisting}[frame=single]
tux63:~/Desktop/RCOM/scripts# route -n
Kernel IP routing table
Destination     Gateway         Genmask         Flags Metric Ref    Use Iface
0.0.0.0         172.16.61.254   0.0.0.0         UG    0      0        0 eth0
172.16.61.0     172.16.61.254   255.255.255.0   UG    0      0        0 eth0
172.16.61.0     0.0.0.0         255.255.255.0   U     0      0        0 eth0
tux63:~/Desktop/RCOM/scripts# traceroute 172.16.60.1
traceroute to 172.16.60.1 (172.16.60.1), 30 hops max, 60 byte packets
 1  172.16.61.254 (172.16.61.254)  0.498 ms  0.548 ms  0.587 ms
 2  172.16.61.253 (172.16.61.253)  0.873 ms  0.500 ms  0.506 ms
 3  172.16.60.1 (172.16.60.1)  0.799 ms  0.792 ms  0.784 ms
tux63:~/Desktop/RCOM/scripts# ping 172.16.60.1
PING 172.16.60.1 (172.16.60.1) 56(84) bytes of data.
64 bytes from 172.16.60.1: icmp_seq=1 ttl=62 time=0.629 ms
64 bytes from 172.16.60.1: icmp_seq=2 ttl=62 time=0.594 ms
64 bytes from 172.16.60.1: icmp_seq=3 ttl=62 time=0.587 ms
64 bytes from 172.16.60.1: icmp_seq=4 ttl=62 time=0.569 ms
64 bytes from 172.16.60.1: icmp_seq=5 ttl=62 time=0.623 ms
^C
--- 172.16.60.1 ping statistics ---
5 packets transmitted, 5 received, 0% packet loss, time 4000ms
rtt min/avg/max/mdev = 0.569/0.600/0.629/0.031 ms
tux63:~/Desktop/RCOM/scripts# traceroute 172.16.60.1
traceroute to 172.16.60.1 (172.16.60.1), 30 hops max, 60 byte packets
 1  172.16.61.253 (172.16.61.253)  0.465 ms  0.343 ms  0.344 ms
 2  172.16.60.1 (172.16.60.1)  0.666 ms  0.662 ms  0.654 ms
tux63:~/Desktop/RCOM/scripts# route -n
Kernel IP routing table
Destination     Gateway         Genmask         Flags Metric Ref    Use Iface
0.0.0.0         172.16.61.254   0.0.0.0         UG    0      0        0 eth0
172.16.61.0     172.16.61.254   255.255.255.0   UG    0      0        0 eth0
172.16.61.0     0.0.0.0         255.255.255.0   U     0      0        0 eth0
tux63:~/Desktop/RCOM/scripts# 
\end{lstlisting}

\chapter{Wireshark logs and statistics}%XXXXXXXXXXXXXXXXXXXXXXXXXXXXXXXXXXXX

\section{Ex1}%XXXXXXXXXXXXXXXXXXXXXXXXXXXXXXXXXXXX
\label{ex1_headers}
\subsection{Captura no tux1 - ARP}
\includegraphics[width=18cm]{ex1_arp.png}
\subsection{Captura no tux1 - ICMP}
\includegraphics[width=18cm]{ex1_icmp.png}

\section{Ex2}%XXXXXXXXXXXXXXXXXXXXXXXXXXXXXXXXXXXX
\label{ex_2}
\subsection{Alínea 7 - Captura no tux1}
\includegraphics[width=18cm]{ex2_a7_tux1.png}
\subsection{Alínea 7 - Captura no tux2}
\label{ex2_tux1ping_tux2}
\includegraphics[width=18cm]{ex2_a7_tux2.png}
\subsection{Alínea 7 - Captura no tux4}
\label{ex2_tux1ping_tux4}
\includegraphics[width=18cm]{ex2_a7_tux4.png}

\subsection{Alínea10 - Captura no tux1}
\includegraphics[width=18cm]{ex2_a10_tux1.png}
\subsection{Alínea10 - Captura no tux2}
\includegraphics[width=18cm]{ex2_a10_tux2.png}
\subsection{Alínea10 - Captura no tux4}
\includegraphics[width=18cm]{ex2_a10_tux4.png}

\section{Ex3}%XXXXXXXXXXXXXXXXXXXXXXXXXXXXXXXXXXXX

\subsection{Capturas no tux4.eth0}
\label{ex3_eth0}
\includegraphics[width=18cm]{ex3_tux4eth0_ARP.png}

\subsection{Capturas no tux4.eth1}
\label{ex3_eth1}
\includegraphics[width=18cm]{ex3_tux4eth1_pingdetailed.png}

\section{Ex4}%XXXXXXXXXXXXXXXXXXXXXXXXXXXXXXXXXXXX

\section{Ex5}%XXXXXXXXXXXXXXXXXXXXXXXXXXXXXXXXXXXX

\section{Ex6}%XXXXXXXXXXXXXXXXXXXXXXXXXXXXXXXXXXXX

\subsection{Capturas dos `hanshakes' no tux1}
\label{ex6_tux1_handshakes}
\includegraphics[width=18cm]{ex6_tux1_handshakes.png}

\subsection{Primeiro `Handshake' no tux1 em Detalhe}
\label{ex6_tux1_1sthandshake}
\includegraphics[width=18cm]{ex6_handshake1.png}
\includegraphics[width=18cm]{ex6_handshake2.png}
\includegraphics[width=18cm]{ex6_handshake3.png}

\subsection{Capturas no tux1 de Dup ACK, Fast Retransmission e Retransmission}
\label{ex6_retrans}
\includegraphics[width=18cm]{ex6_tux1_3054.png}
\includegraphics[width=18cm]{ex6_tux1_fastretransmission.png}
\includegraphics[width=18cm]{ex6_tux1_retransmission.png}

\subsection{Alínea 5 - Gráfico de Tráfego no tux1}
\label{ex6_a5_1io}
\includegraphics[width=18cm]{ex6_a5_tux1_IO.png}
\subsection{Alínea 5 - Gráfico de Tráfego no tux2}
\label{ex6_a5_2io}
\includegraphics[width=18cm]{ex6_a5_tux2_IO.png}

\section{Ex7}%XXXXXXXXXXXXXXXXXXXXXXXXXXXXXXXXXXXX

\subsection{Capturas TCP no tux4.eth0 e tux4.eth1}
\label{ex7_tcp}
\includegraphics[width=18cm]{ex7_TCP.png}
\subsection{Captura UDP e timeout no tux4.eth0}
\label{ex7_udp}
\includegraphics[width=18cm]{ex7_udp_timeout.png}
\subsection{Capturas TCP no tux4.eth0}
\label{ex7_ping}
\includegraphics[width=18cm]{ex7_ping.png}

%%%%%%%%%%%%%%%%%%%%%%%%%%%
% APENDICE - CODIGO FONTE %
%%%%%%%%%%%%%%%%%%%%%%%%%%%
\chapter{Código Fonte}

%%%%%%%%%%%%%%%%%%%%%%%%%%%
\section{downloader.c}
\label{DOWNLOADERC}
%%%%%%%%%%%%%%%%%%%%%%%%%%%

\begin{lstlisting}

#include <string.h>
#include <sys/socket.h>
#include <netinet/in.h>
#include <arpa/inet.h>
#include <netdb.h>
#include <stdlib.h>
#include <unistd.h>
#include <stdio.h>
#include <string.h>
#include <errno.h> 
#include <sys/types.h>
#include "utilities.h"
#include "ftp.h"

//VARS AND STRUCTS --------------------------------------------------------------------------------------
#define FTP_PORT	21
#define MAX_STRING_SIZE 200
struct /*???*/Info{
  char username[MAX_STRING_SIZE];
  char password[MAX_STRING_SIZE];
  char host_name[MAX_STRING_SIZE];
  char url_path[MAX_STRING_SIZE];
  char filename[MAX_STRING_SIZE];
  char ip[MAX_STRING_SIZE];
};

//AUX FUNCS CODE --------------------------------------------------------------------------------------
int parse(char *str, struct Info* info) {
  
  //http://docs.roxen.com/pike/7.0/tutorial/strings/sscanf.xml
	if(4 != sscanf(str, "ftp://[%[^:]:%[^@]@]%[^/]/%s\n", info->username, info->password, info->host_name, info->url_path)) {
		return 1;
	}

  //get filename http://stackoverflow.com/questions/32822988/get-the-last-token-of-a-string-in-c
      char *last = strrchr(info->url_path, '/') ;
      if(last!=NULL) 
      {
	memcpy(info->filename, last+1, strlen(last)+1);
	memset(last,0,strlen(last)+1);
      }
      else {
	strcpy(info->filename,info->url_path);
	memset(info->url_path,0,sizeof(info->url_path));
      }
  
	return 0;
}

int get_ip(struct Info* info) {
	struct hostent* host;

	if ((host = gethostbyname(info->host_name)) == NULL) {
		perror("gethostbyname");
		return 1;
	}

	char* ip = inet_ntoa(*((struct in_addr *)host->h_addr));
	strcpy(info->ip, ip);

	printf("Host name  : %s\n", host->h_name);
	printf("IP Address : %s\n", info->ip);
	
	return 0;
}


//MAIN --------------------------------------------------------------------------------------
#define DEBUG_ALL 1
int main(int argc,char **argv)
{
  struct Info info;
  
  // ftp message composition: ftp://[<user>:<password>@]<host>/<url-path>
  
	// ---- URL stuff ----

	//parse
	if(parse(argv[1],&info)!=OK)
	{
	  printf("\nINVALID ARGUMENT! couldn't be parsed properly.\n");
	  return 1;
	}
	DEBUG_SECTION(DEBUG_ALL,
	printf("\nuser:%s\n",info.username);
	printf("pass:%s\n",info.password);
	printf("host:%s\n",info.host_name);
	printf("urlpath:%s\n",info.url_path);
	printf("filename:%s\n",info.filename);
	);
  
	//- - - - - -
	get_ip(&info);
	
	// ---- FTP stuff -----
		
printf("\n connecting... \n");
	
	if(ftp_connect(info.ip, FTP_PORT)!=OK)
{ftp_abort(); return 1;}

printf("\n logging in... \n");

	if(ftp_login(info.username, info.password)!=OK)// Send user n pass
{ftp_abort(); return 1;}
		

		
	if(strlen(info.url_path)>0) {
	  printf("\n changing dir... \n");
	  
	  if(ftp_changedir(info.url_path)!=OK)// change directory
	  {ftp_abort(); return 1;}
	}
	
printf("\n passive mode... \n");

	if(ftp_pasv()!=OK)// passive mode
{ftp_abort(); return 1;}

printf("\n asking for file... \n");

	if(ftp_retr(info.filename)!=OK)// ask to receive file
{ftp_abort(); return 1;}

printf("\n downloading file... \n");

	if(ftp_download(info.filename)!=OK)// receive file
{ftp_abort(); return 1;}

printf("\n disconecting... \n");

	if(ftp_disconnect()!=OK)// disconnect from server
{ftp_abort(); return 1;}

printf("\n downloader terminated ok! \n");

	return 0;
}

\end{lstlisting}
%%%%%%%%%%%%%%%%%%%%%%%%%%%
\section{Ficheiro ftp.h}
\label{FTPH}
%%%%%%%%%%%%%%%%%%%%%%%%%%%

\begin{lstlisting}

#ifndef FTP
#define FTP

int ftp_connect( const char* ip, int port);
int ftp_disconnect();

int ftp_login( const char* user, const char* password);
int ftp_changedir( const char* path);
int ftp_pasv();
int ftp_retr( const char* filename);
int ftp_download( const char* filename);

void ftp_abort();

#endif

\end{lstlisting}
%%%%%%%%%%%%%%%%%%%%%%%%%%%
\section{Ficheiro ftp.c}
\label{FTPC}
%%%%%%%%%%%%%%%%%%%%%%%%%%%

\begin{lstlisting}

#include <stdio.h>
#include <unistd.h>
#include <string.h>

#include <sys/types.h>
#include <sys/socket.h>


#include "ftp.h"
#include "socket.h"
#include "utilities.h"

#define MAX_STRING_SIZE 500

int control_socket_fd; 
int data_socket_fd;


//------------------------------------------------------------------------
// READ AND SEND
#if 1

int ftp_read(char* str,unsigned long str_total_size)
{
    int bytes = 0;
    if( (bytes = recv(control_socket_fd,str,str_total_size,0)) < 0  )
      {
	perror("ftp_read: recv failed\n");
	return -1;
      }
    return bytes;
}

int ftp_send( const char* str,unsigned long str_size)
{
	    int bytes = 0;
    if( (bytes = send(control_socket_fd,str,str_size,0)) < 0  )
      {
	perror("ftp_read: recv failed\n");
	return -1;
      }
    return bytes;
}

#endif


//------------------------------------------------------------------------
// CONECT AND DISCONECT
#if 1

int ftp_connect( const char* ip, int port) {
	
	int socket_fd;
	char read_bytes[MAX_STRING_SIZE];

	//open control socket
	if ((socket_fd = connect_socket_TCP(ip, port)) < 0)
	{
		printf("ftp_connect: Failed to connect socket\n");
		return 1;
	}

	control_socket_fd = socket_fd;
	data_socket_fd 	  = 0;

	//Try to read with control socket
	if (ftp_read(read_bytes, sizeof(read_bytes))<0)
	{
		printf("ftp_connect: Failed to read\n");
		return 1;
	}

	return 0;
}

int ftp_disconnect() {
	char aux[MAX_STRING_SIZE];

	//read discnnect
		if (ftp_read(aux, sizeof(aux))<0) {
		printf("ftp_disconnect: Failed to disconnect\n");
		return 1;
	}
	//send disconnect 
	sprintf(aux, "QUIT\r\n");
	if (ftp_send(aux, strlen(aux))<0) {
		printf("ftp_disconnect: Failed to output QUIT");
		return 1;
	}
	
	close(control_socket_fd);

	return 0;
}

#endif

//------------------------------------------------------------------------
// MAIN OPERATIONS
#if 1

int ftp_login( const char* user, const char* password) {
	
	char aux[MAX_STRING_SIZE];

	//send username
	sprintf(aux, "user %s\r\n", user);
	if (ftp_send( aux, strlen(aux))< 0) {
		printf("ftp_login: ftp_send failure.\n");
		return 1;
	}
	//receive answer to username
	if (ftp_read( aux, sizeof(aux))<0) {
		printf(	"ftp_login:Bad response to user\n");
		return 1;
	}

	//send password
	memset(aux, 0, sizeof(aux));//reuse 2send
	sprintf(aux, "pass %s\r\n", password);
	if (ftp_send( aux, strlen(aux))< 0) {
		printf("ftp_login: failed to send password.\n");
		return 1;
	}
	//receive answer to password
	if (ftp_read( aux, sizeof(aux))<0) 
	{
		printf(	"ftp_login:Bad response to pass\n");
		return 1;
	}

	return 0;
}

int ftp_changedir(const char* path) {
	
	char aux[MAX_STRING_SIZE];

	//send cwd command
	sprintf(aux, "CWD %s\r\n", path);
	if (ftp_send(aux, strlen(aux))< 0) {
		printf("ftp_changedir:Failed to send\n");
		return 1;
	}

	//get response
	if (ftp_read(aux, sizeof(aux))< 0) {
		printf("ftp_changedir:Failed to get a valid response\n");
		return 1;
	}

	return 0;
}

#define DEBUG_PASV 1
int ftp_pasv() {

	char aux[MAX_STRING_SIZE] = "PASV\r\n";
	
	//send pasv msg
	if (ftp_send(aux, strlen(aux))< 0) {
		printf("ftp_pasv: Failed to enter in passive mode\n");
		return 1;
	}
	
	//receive response
	if (ftp_read(aux, sizeof(aux))<0) {
		printf("ftp_pasv: Failed to receive information to enter passive mode\n");
		return 1;
	}

		DEBUG_SECTION(DEBUG_PASV,printf("pasv():received:%s\n",aux);
	);
	
	// info was received. scan it
	int ip_bytes[4];
	int ports[2];
		
	if ((sscanf(aux, "%*[^(](%d,%d,%d,%d,%d,%d)",
	ip_bytes,&ip_bytes[1], &ip_bytes[2], &ip_bytes[3], ports, &ports[1])) 
		!=6 ) 
	{
		printf("ftp_pasv: Cannot process received data, must receive 6 bytes\n");
		return 1;
	}
	
	// reuse aux and get ip 
	memset(aux, 0, sizeof(aux));
	if ((sprintf(aux, "%d.%d.%d.%d",
	ip_bytes[0], ip_bytes[1], ip_bytes[2], ip_bytes[3]))
		<7) 
	{
		printf("ftp_pasv: Cannot compose ip address\n");
		return 1;
	}

		DEBUG_SECTION(DEBUG_PASV,printf("pasv():ip:%s\n",aux);
	);
	
	// calculate port
	int portResult = ports[0] * 256 + ports[1];

	printf("IP: %s\n", aux);
	printf("PORT: %d\n", portResult);

	if ((data_socket_fd = connect_socket_TCP(aux, portResult)) < 0) {
		printf(	"ftp_pasv: Failed to connect data socket\n");
		return 1;
	}

	return 0;
}

#define DEBUG_RETR 1
int ftp_retr(const char* filename) {
	char aux[MAX_STRING_SIZE];

	//send retr
	sprintf(aux, "RETR %s\r\n", filename);
	//sprintf(aux, "LIST %s\r\n", "");
	if (ftp_send(aux, strlen(aux))< 0) {
		printf("ftp_retr: Failed to send \n");
		return 1;
	}

	//get respones
	if (ftp_read(aux, sizeof(aux))< 0) {
		printf("ftp_retr: Failed to get response\n");
		return 1;
	}
	
	DEBUG_SECTION(DEBUG_PASV,printf("ftp_retr_debug_1:%s\n",aux););
	
	return 0;
}

#define DEBUG_DOWNLOAD 0
int ftp_download(const char* filename) {
	
  printf("\ndata_%d__cont_%d\n",data_socket_fd,  control_socket_fd);

	FILE* file;
	int bytes;

	//create n open file
	if (!(file = fopen(filename, "w"))) {
		printf("ftp_download: Failed to create/open file\n");
		return 1;
	}


	char buf[MAX_STRING_SIZE];
	while ((bytes = recv(data_socket_fd,buf,MAX_STRING_SIZE,0))>0) {
		if (bytes < 0) {
			perror("ftp_download: Failed to receive from data socket\n");
			fclose(file);
			return 1;
		}
		
		DEBUG_SECTION(DEBUG_DOWNLOAD,
			      printf("bytes:%d\n",bytes);
		              printf("rec:%s\n",buf);
			     );
		
		//output received bytes to file
		if ((bytes = fwrite(buf, bytes, 1, file)) < 0) {
			perror("ftp_download: Failed to write data in file\n");
			return 1;
		}
	}

	//close file and data socket
	fclose(file);
	close(data_socket_fd);

	return 0;
}

void ftp_abort()
{
	printf("\n ABORTED! \n");
	if(data_socket_fd) close(data_socket_fd);
	if(control_socket_fd) close(control_socket_fd);
	
}

#endif


\end{lstlisting}
%%%%%%%%%%%%%%%%%%%%%%%%%%%
\section{socket.h}
\label{SOCKETH}
%%%%%%%%%%%%%%%%%%%%%%%%%%%

\begin{lstlisting}

#ifndef SOCKET
#define SOCKET

/*return socket fd*/
int connect_socket_TCP(const char* ip, int port);

#endif

\end{lstlisting}
%%%%%%%%%%%%%%%%%%%%%%%%%%%
\section{socket.c}
\label{SOCKETC}
%%%%%%%%%%%%%%%%%%%%%%%%%%%

\begin{lstlisting}

#include <sys/socket.h>
#include <netinet/in.h>
#include <arpa/inet.h>
//#include <netdb.h>
#include <strings.h>
#include <stdio.h>


#include "socket.h"

int connect_socket_TCP(const char* ip, int port)
{
	//adapted from clientTCP.c
	
	int socket_fd;
	struct sockaddr_in server_addr;

	// server address handling
	bzero((char*) &server_addr, sizeof(server_addr));
	server_addr.sin_family = AF_INET;
	server_addr.sin_addr.s_addr = inet_addr(ip); /*32 bit Internet address network byte ordered*/
	server_addr.sin_port = htons(port); /*server TCP port must be network byte ordered */

	// open an TCP socket
	if ((socket_fd = socket(AF_INET, SOCK_STREAM, 0)) < 0) {
		perror("connect_socket:socket()");
		return -1;
	}

	// connect to the server
	if (connect(socket_fd, (struct sockaddr *) &server_addr, sizeof(server_addr)) < 0) {
		perror("connect_socket:connect()");
		return -1;
	}

	return socket_fd;
}

\end{lstlisting}
\pagebreak
\input{./utilitiesh.tex}

\end{appendices}

\end{document}
