%%%%%%%%%%%%%%%%%%%%%%%%%%%
\section{Ficheiro downloader.c}
\label{DOWNLOADERC}
%%%%%%%%%%%%%%%%%%%%%%%%%%%

\begin{lstlisting}

#include <string.h>
#include <sys/socket.h>
#include <netinet/in.h>
#include <arpa/inet.h>
#include <netdb.h>
#include <stdlib.h>
#include <unistd.h>
#include <stdio.h>
#include <string.h>
#include <errno.h> 
#include <sys/types.h>
#include "utilities.h"
#include "ftp.h"

//VARS AND STRUCTS --------------------------------------------------------------------------------------
#define FTP_PORT	21
#define MAX_STRING_SIZE 200
struct /*???*/Info{
  char username[MAX_STRING_SIZE];
  char password[MAX_STRING_SIZE];
  char host_name[MAX_STRING_SIZE];
  char url_path[MAX_STRING_SIZE];
  char filename[MAX_STRING_SIZE];
  char ip[MAX_STRING_SIZE];
};

//AUX FUNCS CODE --------------------------------------------------------------------------------------
int parse(char *str, struct Info* info) {
  
  //http://docs.roxen.com/pike/7.0/tutorial/strings/sscanf.xml
	if(4 != sscanf(str, "ftp://[%[^:]:%[^@]@]%[^/]/%s\n", info->username, info->password, info->host_name, info->url_path)) {
		return 1;
	}

  //get filename http://stackoverflow.com/questions/32822988/get-the-last-token-of-a-string-in-c
      char *last = strrchr(info->url_path, '/') ;
      if(last!=NULL) 
      {
	memcpy(info->filename, last+1, strlen(last)+1);
	memset(last,0,strlen(last)+1);
      }
      else {
	strcpy(info->filename,info->url_path);
	memset(info->url_path,0,sizeof(info->url_path));
      }
  
	return 0;
}

int get_ip(struct Info* info) {
	struct hostent* host;

	if ((host = gethostbyname(info->host_name)) == NULL) {
		perror("gethostbyname");
		return 1;
	}

	char* ip = inet_ntoa(*((struct in_addr *)host->h_addr));
	strcpy(info->ip, ip);

	printf("Host name  : %s\n", host->h_name);
	printf("IP Address : %s\n", info->ip);
	
	return 0;
}


//MAIN --------------------------------------------------------------------------------------
#define DEBUG_ALL 1
int main(int argc,char **argv)
{
  struct Info info;
  
  // ftp message composition: ftp://[<user>:<password>@]<host>/<url-path>
  
	// ---- URL stuff ----

	//parse
	if(parse(argv[1],&info)!=OK)
	{
	  printf("\nINVALID ARGUMENT! couldn't be parsed properly.\n");
	  return 1;
	}
	DEBUG_SECTION(DEBUG_ALL,
	printf("\nuser:%s\n",info.username);
	printf("pass:%s\n",info.password);
	printf("host:%s\n",info.host_name);
	printf("urlpath:%s\n",info.url_path);
	printf("filename:%s\n",info.filename);
	);
  
	//- - - - - -
	get_ip(&info);
	
	// ---- FTP stuff -----
		
printf("\n connecting... \n");
	
	if(ftp_connect(info.ip, FTP_PORT)!=OK)
{ftp_abort(); return 1;}

printf("\n logging in... \n");

	if(ftp_login(info.username, info.password)!=OK)// Send user n pass
{ftp_abort(); return 1;}
		

		
	if(strlen(info.url_path)>0) {
	  printf("\n changing dir... \n");
	  
	  if(ftp_changedir(info.url_path)!=OK)// change directory
	  {ftp_abort(); return 1;}
	}
	
printf("\n passive mode... \n");

	if(ftp_pasv()!=OK)// passive mode
{ftp_abort(); return 1;}

printf("\n asking for file... \n");

	if(ftp_retr(info.filename)!=OK)// ask to receive file
{ftp_abort(); return 1;}

printf("\n downloading file... \n");

	if(ftp_download(info.filename)!=OK)// receive file
{ftp_abort(); return 1;}

printf("\n disconecting... \n");

	if(ftp_disconnect()!=OK)// disconnect from server
{ftp_abort(); return 1;}

printf("\n downloader terminated ok! \n");

	return 0;
}

\end{lstlisting}