%%%%%%%%%%%%%%%%%%%%%%%%%%%%%%%%%%%%%%%%%%%%%%%%%%%%%
%												    %
%	PROTOCOLO LIGACAO DADOS						    %
%												    %
%	Novembro 2015								    %
%												    %
%	Angela Cardodo e Bruno Madeira					%
%   											    %	
%%%%%%%%%%%%%%%%%%%%%%%%%%%%%%%%%%%%%%%%%%%%%%%%%%%%%

\documentclass[11pt,a4paper,reqno]{report}
\linespread{1.5}

\usepackage[active]{srcltx}    
\usepackage{graphicx}
\usepackage{amsthm,amsfonts,amsmath,amssymb,indentfirst,mathrsfs,amscd}
\usepackage[mathscr]{eucal}
\usepackage[active]{srcltx} %inverse search
\usepackage{tensor}
\usepackage[utf8x]{inputenc}
\usepackage[portuges]{babel}
\usepackage[T1]{fontenc}
\usepackage{enumitem}
\setlist{nolistsep}
\usepackage{comment} 
\usepackage{tikz}
\usepackage[numbers,square, comma, sort&compress]{natbib}
\usepackage[nottoc,numbib]{tocbibind}
%\numberwithin{figure}{section}
\numberwithin{equation}{section}
\usepackage{scalefnt}
\usepackage[top=2.5cm, bottom=2.5cm, left=2.5cm, right=2.5cm]{geometry}
%\usepackage{tweaklist}
%\renewcommand{\itemhook}{\setlength{\topsep}{0pt}%
%	\setlength{\itemsep}{0pt}}
%\renewcommand{\enumhook}{\setlength{\topsep}{0pt}%
%	\setlength{\itemsep}{0pt}}
%\usepackage[colorlinks]{hyperref}
\usepackage{MnSymbol}
%\usepackage[pdfpagelabels,pagebackref,hypertexnames=true,plainpages=false,naturalnames]{hyperref}
\usepackage[naturalnames]{hyperref}
\usepackage{enumitem}
\usepackage{titling}
\newcommand{\subtitle}[1]{%
	\posttitle{%
	\par\end{center}
	\begin{center}\large#1\end{center}
	\vskip0.5em}%
}
\newcommand{\HRule}{\rule{\linewidth}{0.5mm}}
\usepackage{caption}
\usepackage{etoolbox}% http://ctan.org/pkg/etoolbox
\usepackage{complexity}

\usepackage[official]{eurosym}

\def\Cpp{C\raisebox{0.5ex}{\tiny\textbf{++}}}

\makeatletter
\def\@makechapterhead#1{%
  %%%%\vspace*{50\p@}% %%% removed!
  {\parindent \z@ \raggedright \normalfont
    \ifnum \c@secnumdepth >\m@ne
        \huge\bfseries \@chapapp\space \thechapter
        \par\nobreak
        \vskip 20\p@
    \fi
    \interlinepenalty\@M
    \Huge \bfseries #1\par\nobreak
    \vskip 40\p@
  }}
\def\@makeschapterhead#1{%
  %%%%%\vspace*{50\p@}% %%% removed!
  {\parindent \z@ \raggedright
    \normalfont
    \interlinepenalty\@M
    \Huge \bfseries  #1\par\nobreak
    \vskip 40\p@
  }}
\makeatother

\usepackage[toc,page]{appendix}

\addto\captionsportuges{%
  \renewcommand\appendixname{Anexo}
  \renewcommand\appendixpagename{Anexos}
}

\addto\captionsportuges{%
  \renewcommand\abstractname{Sumário}
}

\begin{document}



\begin{titlepage}
\begin{center}
 
\vspace*{3cm}

{\Large Redes de Computadores}\\[2cm]

% Title
{\Huge \bfseries Redes de Computadores  \\[1cm]}

% Author
{\large \^Angela Cardoso e Bruno Madeira}\\[2cm]

\includegraphics[width=10cm]{feup_logo.jpg}\\[2cm]


% Bottom of the page
{\large \today}

\end{center}
\end{titlepage}

%%%%%%%%%%%
% SUMARIO %
%%%%%%%%%%%
\begin{abstract}

\end{abstract}

\tableofcontents

%%%%%%%%%%%%%%
% INTRODUCAO %
%%%%%%%%%%%%%%
\chapter{Introdução}

	Relatório relativo ao primeiro trabalho prático de Redes de Computadores que consiste na implementação de uma aplicação que transfere imagens entre dois computadores fazendo uso da porta-série. A aplicação deve usar um protocolo de ligação de dados {\it Stop N Wait ARQ}  híbrido que deve assegurar a fiabilidade  da transmissão mesmo em caso de desconexão. Deve também usar um protocolo de aplicação que é responsável pelo envio da imagem. O código desenvolvido deve ser estruturado em camadas, respeitando o princípio de encapsulamento, de modo a assegurar cada protocolo funciona de forma independente.
	
	O trabalho foi utilizando a linguagem de programação C num ambiente com um sistema operativo baseado em Linux. Para testes foi usada uma porta-seire XPTO???.
	
	Este relatório visa reportar qual o estado final da aplicação desenvolvia, clarificar detalhes do processo de implementação/código e a opinião dos estudantes face ao projecto realizado.
	
	Do capítulo 2 ao 4 são expostas as estruturas e os mecanismos implementados na concepção da aplicação sem incidir em detalhes específicos destes.
	
	Os capítulos 5 e 6 incidem sobre as particularidades da implementação dos protocolos usados.
	
	O capítulo 7 apresenta os testes efectuados sobre a aplicação.
	
	O 

%%%%%%%%%%%%%%%%%%%%
% ESTRUTURA CODIGO %
%%%%%%%%%%%%%%%%%%%%
\chapter{Estrutura do código}

enumerar ficheiros e estruturas

%%%%%%%%%%%%%
% CASOS USO %
%%%%%%%%%%%%%
\chapter{Casos de uso principais}

???

seq. de chamadas a funçoes (talvez fzr aqueles diagramas com o graphviz)

%%%%%%%%%%%%%%%%%%
% LIGACAO LOGICA %
%%%%%%%%%%%%%%%%%%
\chapter{Protocolo de ligação lógica}

ligação de dados assíncrona
vmin=0 e ???min=0 -> read retorna de imediato -> e usado um ciclo -> blablabla

estrutura datalink? guarda as configuraçoes a usar

funcao statemachine trata dos estados em qlq funcao llXPTO mas nao e autonoma, quem a chama tem q que definir qual os estados finais e tratar do reset quando um estado final e atingido mas a funcao nao e terminad.

paridade usada foi XPTO, e possivel alterala no codigo

alarm

llopen()

llread()

llwrite()

llclose() pde tratar do disk ou nao dependendo de como for chamada

%%%%%%%%%%%%%
% APLICACAO %
%%%%%%%%%%%%%
\chapter{Protocolo de aplicação}

foi implementada de acordo com o enunciado.

a aplicação guarda o tamanho de imagem e o numero de pacotes ja enviados/recebidos

envia Start
recebe Start

envia info pode enviar pacotes de tamanho maximo definido pelo utilizador
recebe info

envia end
recebe end

%%%%%%%%%%%%%
% VALIDACAO %
%%%%%%%%%%%%%
\chapter{Validação}

foram realizados testes nas aulas práticas e fora destas usando uma porta-série.

foram realizados testes nas máquinas virtuais usadas para desenvolver a aplicação.

??? +

%%%%%%%%%%%%%%%
% VALORIZACAO %
%%%%%%%%%%%%%%%
\chapter{Elementos de valorização}

registo de ocurrencias

opcoes para o utilizador-
bitrate,
tamanho de info a mandar,
numero de tentativas de reconexao

geraçao de erros

interface - tem um intervalo minimo de atualizacao de modo a evitar interferir com a recepçao de dados

%%%%%%%%%%%%%%
% CONCLUSOES %
%%%%%%%%%%%%%%
\chapter{Conclusões}

%%%%%%%%%%%%%%%%
% BIBLIOGRAPHY %
%%%%%%%%%%%%%%%%
\bibliographystyle{IEEEtran}
\bibliography{rabb,refs}

\begin{appendices}

%%%%%%%%%%%%%%%%%%%%%%%%%%%
% APENDICE - CODIGO FONTE %
%%%%%%%%%%%%%%%%%%%%%%%%%%%
\chapter{Código Fonte}

\end{appendices}

\end{document}
