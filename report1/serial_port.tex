%%%%%%%%%%%%%%%%%%%%%%%%%%%%%%%%%%%%%%%%%%%%%%%%%%%%%
%												    %
%	PROTOCOLO LIGACAO DADOS						    %
%												    %
%	Novembro 2015								    %
%												    %
%	Angela Cardodo e Bruno Madeira					%
%   											    %	
%%%%%%%%%%%%%%%%%%%%%%%%%%%%%%%%%%%%%%%%%%%%%%%%%%%%%

\documentclass[11pt,a4paper,reqno]{report}
\linespread{1.2}


\usepackage{rotating}
\usepackage{tikz}
\usepackage[active]{srcltx}    
\usepackage{graphicx}
\usepackage{amsthm,amsfonts,amsmath,amssymb,indentfirst,mathrsfs,amscd}
\usepackage[mathscr]{eucal}
\usepackage{tensor}
\usepackage[utf8x]{inputenc}
\usepackage[portuges]{babel}
\usepackage[T1]{fontenc}
\usepackage{enumitem}
\setlist{nolistsep}
\usepackage{comment} 
\usepackage{tikz}
\usepackage[numbers,square, comma, sort&compress]{natbib}
\usepackage[nottoc,numbib]{tocbibind}
%\numberwithin{figure}{section}
\numberwithin{equation}{section}
\usepackage{scalefnt}
\usepackage[top=2cm, bottom=2.5cm, left=2cm, right=2cm]{geometry}
\usepackage{MnSymbol}
%\usepackage[pdfpagelabels,pagebackref,hypertexnames=true,plainpages=false,naturalnames]{hyperref}
\usepackage[naturalnames]{hyperref}
\usepackage{enumitem}
\usepackage{titling}
\newcommand{\subtitle}[1]{%
	\posttitle{%
	\par\end{center}
	\begin{center}\large#1\end{center}
	\vskip0.5em}%
}
\newcommand{\HRule}{\rule{\linewidth}{0.5mm}}
\usepackage{caption}
\usepackage{etoolbox}% http://ctan.org/pkg/etoolbox
\usepackage{complexity}

\usepackage[official]{eurosym}

\def\Cpp{C\raisebox{0.5ex}{\tiny\textbf{++}}}

\makeatletter
\def\@makechapterhead#1{%
  \vspace*{2\p@}%
  {\parindent \z@ \raggedright \normalfont
    \ifnum \c@secnumdepth >\m@ne
%        \huge\bfseries \@chapapp\space \thechapter
        \par\nobreak
        \vskip 5\p@
    \fi
    \interlinepenalty\@M
    \Huge \bfseries \thechapter\space #1\par\nobreak
    \vskip 20\p@
  }}
\def\@makeschapterhead#1{%
  \vspace*{2\p@}%
  {\parindent \z@ \raggedright
    \normalfont
    \interlinepenalty\@M
    \Huge \bfseries  #1\par\nobreak
%    \vskip 5\p@
  }}
\makeatother

\usepackage[toc,page]{appendix}

\addto\captionsportuges{%
  \renewcommand\appendixname{Anexo}
  \renewcommand\appendixpagename{Anexos}
  \renewcommand\appendixtocname{Anexos}
  \renewcommand\abstractname{\huge Sumário}  
}

\usepackage{verbatim}
\usepackage{color}
\definecolor{darkgray}{rgb}{0.41, 0.41, 0.41}
\definecolor{green}{rgb}{0.0, 0.5, 0.0}
\usepackage{listings}
\lstset{language=C++, 
	numbers=left,
	firstnumber=1,
	numberfirstline=true,
    basicstyle=\linespread{0.8}\ttfamily,
    keywordstyle=\color{blue}\ttfamily,
	showstringspaces=false,
    stringstyle=\color{red}\ttfamily,
    commentstyle=\color{green}\ttfamily,
	identifierstyle=\color{darkgray}\ttfamily,
    morecomment=[l][\color{magenta}]{\#},
	tabsize=4,
    breaklines=true
}

\begin{document}



\begin{titlepage}
\begin{center}
 
\vspace*{3cm}

{\Large Redes de Computadores}\\[2cm]

% Title
{\Huge \bfseries Redes de Computadores  \\[1cm]}

% Author
{\large \^Angela Cardoso e Bruno Madeira}\\[2cm]

\includegraphics[width=10cm]{feup_logo.jpg}\\[2cm]


% Bottom of the page
{\large \today}

\end{center}
\end{titlepage}


%%%%%%%%%%%
% SUMARIO %
%%%%%%%%%%%
\begin{abstract}
	
Relatório relativo ao primeiro projeto da unidade curricular Redes de Computadores do curso Mestrado Integrado em Engenharia Informática e Computação. O projeto consiste na implementação de uma aplicação que transfere imagens entre dois computadores fazendo uso da porta-série. O objectivo é colocar em prática alguns dos conceitos leccionados na unidade curricular relativos a protocolos de ligação de dados.

Este documento apresenta o estado final do projecto, assim como as considerações dos estudantes responsáveis pela sua implementação face ao resultado obtido.

\end{abstract}

\tableofcontents

%%%%%%%%%%%%%%
% INTRODUCAO %
%%%%%%%%%%%%%%
\chapter{Introdução}

No âmbito da unidade curricular Redes de Computadores, do Mestrado Integrado em Engenharia Informática, foi-nos proposta a realização de um projecto laboratorial, que consiste na implementação de uma aplicação que transfere imagens entre dois computadores fazendo uso da porta-série. 

A aplicação usa o protocolo de ligação de dados \emph{Stop N Wait ARQ} híbrido, que deve assegurar a fiabilidade  da emissão mesmo em caso de desconexão. É também usado um protocolo de aplicação que é responsável pelo envio da imagem. O código desenvolvido está estruturado em camadas, respeitando o princípio de encapsulamento, de modo a assegurar que cada protocolo funciona de forma independente.
	
O projeto utiliza a linguagem de programação C num ambiente com um sistema operativo baseado em Linux. Durante o desenvolvimento foram utilizadas máquinas virtuais, que simulam a ligação da porta de série, e máquinas reais, com uma ligação de porta de série por cabo. O código pode ser verificado em anexo e será referenciado em algumas secções do relatório.
	
Este relatório tem como objectivo reportar qual o estado final da aplicação desenvolvia, clarificar detalhes do processo de implementação/código e a opinião dos estudantes face ao projecto realizado. No Capítulo 2 são expostas as estruturas e os mecanismos implementados na concepção da aplicação numa perspectiva macro. Os detalhes relativos à implementação dos protocolos são apresentados nos Capítulos 3 e 4. Os detalhes relativos à implementação de componentes extra são apresentados no Capítulo 5. O Capítulo 6 é relativo à validação e aos testes efectuados.
	
	
%%%%%%%%%%%%%%%
% ARQUITETURA %
%%%%%%%%%%%%%%%
\chapter{Arquitetura, Estrutura de Código e Casos de uso}

\section{Arquitectura}

Cada módulo desenvolvido no projecto pode ser identificado pelos \emph{header files} do código fonte que são:
\begin{itemize}
\item \verb|DataLinkProtocol| implementa e disponibiliza as funcionalidades da camada de ligação de dados. 
\item \verb|AppProtocol| implementa e disponibiliza as funcionalidades da camada de aplicação relacionadas com o envio/recepção de pacotes.
\item \verb|user_interface| implementa segmentos e disponibiliza funcionalidades ligadas ao interface de utilizador. 
\item \verb|FileFuncs| disponibiliza funções para leitura e escrita de ficheiros. 
\item \verb|App.c| onde está a função main e são chamadas as principais funções de outro módulos em conjunto com as operações adicionais necessárias. 
\item \verb|utilities.h| foi criada com o intuito de ter métodos, estruturas e funcionalidades úteis a todos os módulos desenvolvidos. O seu uso principal no projecto é auxiliar o debug dos diversos modelos desenvolvidos.
\end{itemize}
Podem verificar-se no diagrama de módulos, disponível no anexo \ref{modulediagram}, as relações entre estes módulos.

%%%%%%%%%%%%%%%%%%%%
% ESTRUTURA CODIGO %
%%%%%%%%%%%%%%%%%%%%
%\chapter{Estrutura do código}
\section{Estrutura do código}

%TODO: APIs, principais estruturas de dados, principais funções e sua relação com a arquitetura
Seguidamente são apresentadas as principais estruturas e funções desenvolvidas em cada módulo. Algumas funções estão omissas, dado que são referidas em maior detalhe nos capítulos relativos aos protocolos implementados.

\subsection{Estruturas de dados}
\begin{itemize}
\item \verb|applicationLayer|: declarada na \verb|App.c| contem informações básicas ao programa, como o seu \emph{status} enquanto emissor ou receptor. 	
\item \verb|linkLayer|: declarada no \verb|DataLinkProtocol.c| serve para guardar as definições básicas da camada de ligação de dados, como qual a porta série a ser usada.	
\item \verb|occurrences_Log|: declarada no \verb|DataLinkProtocol.h| serve para registar ocorrências, como o número de REJs recebidos.	
\end{itemize}

\subsection{Funções privadas da DataLinkProtocol}
\begin{itemize}
\item \verb|genBCC2| e \verb|validateBCC2| tratam respectivamente de gerar e de validar o BCC2.
\item \verb|write_UorS| e \verb|write_I| responsáveis pelo envio de tramas.
%getMessage e explicado no capitulo de protocolo
\item \verb|apply_stuffing| e \verb|apply_destuffing| realizam o stuffing e destuffing dos dados nas tramas I.
\item \verb|update_state_machine| função auxiliar que funciona como máquina de estados.
\item \verb|llopen_receiver|, \verb|llopen_transmitter|, \verb|llclose_receiver| e \verb|llclose_transmitter| ajudam a organizar o código de \verb|llopen| e \verb|llclose|, uma vez que a sua execução difere do receptor para o emissor.
%\item \verb|timeout_alarm_handler|
\item \verb|startAlarm| e \verb|stopAlarm| gerem o uso do alarme em conjunto com as funções que as invocam.
\end{itemize}

\subsection{Funções privadas da AppProtocol.c}
\begin{itemize}
\item \verb|receivePacket| trata da recepção de um pacote que pode ser de controlo ou de dados.
\item \verb|getControlPacket| e \verb|getInfoPacket| responsáveis, respectivamente, pela criação de um pacote de controlo e de dados.
\item \verb|sendControlPacket| e \verb|sendInfoPacket| responsáveis, respectivamente, pela emissão de um pacote de controlo e de dados.
\item \verb|show_progress| responsável por mostrar ao utilizador o estado actual da transferência/recepção do ficheiro.
\end{itemize}

\subsection{Funções disponibilizadas por FileFuncs.h}
\begin{itemize}
\item \verb|getFileBytes| e \verb|save2File| são responsáveis, respectivamente, pela escrita e e leitura dos ficheiros.
\end{itemize}

%not impotant put only if it fits in a single page _>>>\subsection{Funções de user\_interface.c}

\subsection{Funções da App.c}
\begin{itemize}
%nao considerei o connect nem o setPacketSize mt importantes
\item \verb|receiveImage| e \verb|sendImage| chamam as funções do Protocolo de Aplicação com os ajustes necessários, de modo a enviar/receber uma imagem.
\item \verb|config| quando o utilizador termina a selecção de opções no interface, esta função transmite-as para o Protocolo de Ligação de Dados.
\end{itemize}


%%%%%%%%%%%%%
% CASOS USO %
%%%%%%%%%%%%%
%\chapter{Casos de uso principais}
\section{Casos de Uso}

A aplicação desenvolvida deve ser chamada na linha de comandos recebendo como argumentos a porta de série a usar (\verb|/dev/ttyS0| ou \verb|/dev/ttyS4|) e um caracter indicador(\verb|t| ou \verb|r|) se a aplicação deve correr em modo emissor ou receptor.

Uma vez invocado o programa será mostrado um menu ao utilizador. No emissor pode escolher seleccionar uma imagem a enviar ou enviar uma que já tenha seleccionado. Do lado do receptor pode ser iniciada a recepção. Ambas as versões da aplicação têm um submenu de para seleccionar  opções, estas  serão abordadas numa outra secção do relatório.

As funções invocadas no seguimento das escolhas do utilizador podem ser averiguadas no diagrama de chamada de funções que pode ser consultado no anexo \ref{flux}.

%%%%%%%%%%%%%%%%%%
% LIGACAO LOGICA %
%%%%%%%%%%%%%%%%%%
\chapter{Protocolo de ligação lógica}

Para implementar o protocolo de ligação lógica, seguimos as indicações do enunciado do projeto. Sendo assim, usamos a variante \emph{Stop N Wait}, o que significa que o Emissor, após cada mensagem, aguarda uma resposta do Recetor antes de enviar a mensagem seguinte. Isto significa, entre outras coisas, que podemos utilizar uma numeração módulo 2 para as mensagens, dado que nunca temos mais do que 2 mensagens em jogo (aquela que foi enviada e a que se pretende enviar de seguida).

O interface deste protocolo disponibiliza 6 funções. Quatro são as definidas pelo guião do trabalho (\verb|llopen|, \verb|llread|, \verb|llwrite|, \verb|llclose|). As únicas alterações efectuadas foram relativas à assinatura e funcionamento de \verb|llclose|, de modo a permitir fechar a porta-série em caso de erro, e no \verb|llopen|, que não recebe a porta a abrir. 

Estas 4 funções utilizam a função \verb|update_state_machine| para determinar a cada instante o estado em que está.
Com a concepção desta função evitou-se alguma repetição de código, mas foi necessário cuidado extra na implementação pois existe troca de informação entre esta e a função que a invoca. De facto, a função \verb|update_state_machine| recebe o tipo de trama esperada, de forma a identificar a função que a invocou e atualizar os estados corretamente, e guarda numa variável auxiliar, \verb|received_C_type|, o tipo de trama recebida. Quando a função que a invocou detecta que se atingiu o estado \verb|STATE_MACHINE_STOP|, verifica o \verb|received_C_type| para saber qual tipo de trama recebido e como deve proceder.

Para obter a trama que se pretende enviar são usadas as funções \verb|write_UorS|, para envio de tramas de Supervisão ou Não numeradas, e \verb|write_I|, para tramas de Informação. Elas fazem uso da função \verb|getMessage| que tem como parâmetros o Emissor original, o tipo e o número da trama e o array de caracteres onde será colocada a flag e o cabeçalho da trama. Posteriormente a estes bytes, em tramas de Informação, serão adicionados os dados e o respetivo BCC2. Em todas as tramas, é apenas reutilizada a flag que está na primeira posição do array. A sua especificação pode ser encontrada no Anexo~\ref{tramas}.

As outras 2 funções disponibilizadas pelo interface são \verb|set_basic_definitions| e \verb|get_occurrences_log| que servem, respectivamente, para guardar as opções escolhidas pelo utilizador e para receber o registo de ocorrências na camada da aplicação.


%%%%%%%%%%%%%
% APLICACAO %
%%%%%%%%%%%%%
\chapter{Protocolo de aplicação}

O protocolo de aplicação foi implementado no ficheiro \verb|AppProtocol.c|. e apenas disponibiliza as funções \verb|sendFile| e \verb|receiveFile| para módulos externos. Estas funções são responsáveis pelo envio e recepção de uma imagem completa, sendo que a recepção e envio de pacotes individuais é tratada dentro de \verb|AppProtocol.c|.

O protocolo foi implementado de acordo com o enunciado e é responsável pelo envio de pacotes de controlo e de dados.
Os pacotes de controlo são enviados antes e após o envio dos dados e têm como parâmetros o nome e tamanho (em bytes) da imagem.
Quando a informação destes pacotes não coincide, não é realizado o output da imagem, dado não haver garantias que os pacotes de dados pertençam todos à mesma imagem ou que foram todos recebidos.

O envio/recepção de imagens atualiza uma variável, recebida por um apontador, externa ao Protocolo (definida num outro módulo), que indica os bytes já enviados/recebidos. Tal variável é usada atualmente dentro do protocolo para o display do estado de envio da imagem, que inicialmente estava implementado no ficheiro \verb|user_interface.c|. Esta variável serviria também para tentar reenviar uma imagem em caso de um envio falhar a meio. Este mecanismo de reenvio não foi completado e o display foi movido pois o sua implementação inicial interferia com o envio/recepção de dados. A estrutura do código actual reflecte os planos inicialmente concebidos.

%%%%%%%%%%%%%%%
% VALORIZACAO %
%%%%%%%%%%%%%%%
\chapter{Elementos de valorização}

\section{Registo de ocorrências}

A camada de ligação de dados regista o número de tramas do tipo I e REJs recebidas/enviadas e o número de ocorrências de timeouts de uma emissão. A informação fica registada numa variável do tipo \verb|occurrences_Log| e pode ser acedida pela componente de interface através do método \verb|get_occurrences_log|, como pode ser observado na Figura~\ref{logimg} no Anexo~\ref{imagens_app}.

\section{Definições básicas}

O utilizador pode definir qual o \verb|baudrate| a usar, o \verb|packetSize| (numero de bytes máximos que envia da imagem em cada trama I antes de se realizar o stuffing) e o número máximo de tentativas consecutivas de conexão.
Uma vez escolhidas as opções, o intervalo de duração de um timeout é calculado em função do \verb|baudrate| e do \verb|packetSize| escolhidos pelo utilizador, como se pode verificar no anexo \ref{DATALINKPROTOCOLC} nas linhas 187 a 209. Na nossa implementação o receptor pode e deve configurar o \verb|baudrate| e \verb|packetSize| iguais aos do emissor, pois são necessários para calcular um intervalo de timeout que seja fiável face às definições do emissor.

\section{Gerador aleatório de erros e REJs}
São gerados erros aleatoriamente nas tramas enviadas pelo emissor. Estes erros podem ocorrer no cabeçalho ou nos dados da trama, com probabilidades independentes. Para tal, são usadas as funções \verb|randomErrorCabe| e \verb|randomErrorData|. A frequência destes erros é defina a partir das constantes \verb|CABE_PROB| e \verb|DATA_PROB| que representam quantas tramas são enviadas por cada um que apresenta um erro do tipo correspondente. Pode ser verificada a função \verb|randomErrorData| no anexo \ref{DATALINKPROTOCOLC} nas linhas 517 a 526. A função \verb|randomErrorCabe| está no mesmo anexo e é muito semelhante.

Também foram implementadas tramas do tipo REJ. Quando o receptor consegue identificar ocorrência de erro(s) no bloco de dados envia uma mensagem ao emissor e este trata de reenviar a pacote novamente. Por exemplo, os REJ observados na Figura~\ref{logimg} no Anexo~\ref{imagens_app} são resultado de erros aleatórios nos dados. Já os \emph{timeouts}, na mesma figura, são devidos a erros no cabeçalho, dado que neste caso o recetor ignora a trama e o emissor fica à espera de uma resposta que não vem.

\section{Representação do progresso}
Na AppProtocol.c, quando a aplicação está enviar/receber uma imagem, trata de imprimir na consola a quantidade de bytes enviados/recebidos e quantos faltam para enviar/receber a imagem na totalidade. Este \emph{display} (impressão na consola) é actualizado a cada pacote ou após o tempo mínimo definido na constante \verb|UPDATE_DISPLAY_MIN_TIME_INTERVAL|, de modo a evitar que ocorra com elevada frequência, o que poderia tornar difícil a sua leitura ou afectar o envio/receção dos dados. Um exemplo do progresso pode ser observado na Figura~\ref{progressimg} no Anexo~\ref{imagens_app}.

%%%%%%%%%%%%%
% VALIDACAO %
%%%%%%%%%%%%%
\chapter{Validação}

Foram realizados cerca de 10 testes ao longo das aulas práticas e fora destas usando máquinas com uma ligação física por cabo da porta-série. Além disso, realizamos aproximadamente 30 testes nas máquinas virtuais usadas para desenvolver a aplicação quando a implementação já se encontrava pronta para a entrega.

Relativamente à qualidade dos testes não foi usado nenhum auxiliar para verificar se as imagens recebidas eram iguais às enviadas nem nenhuma automatização de testes, sendo estes todos realizados manualmente. Tivemos o cuidado de escolher imagens de tamanhos variados entre os 263 bytes (a mais pequena) e 712K bytes (a maior) e de experimentar diferentes baudrates. Os tipos de imagens também variavam entre os tipos gif, jpeg e png.

Uma vez atingido um nível estável de desenvolvimento da aplicação, todos os testes realizados foram bem sucedidos, à excepção de 1 que apresentou um aviso inesperado e do qual não conseguimos apurar o problema, nem reproduzir.

%%%%%%%%%%%%%%
% CONCLUSOES %
%%%%%%%%%%%%%%
\chapter{Conclusões}

Neste projecto, como acontece frequentemente em projetos desta natureza, há certamente melhorias a efetuar. Em todo o caso, consideramos que a aplicação foi implementada com sucesso dentro do prazo estabelecido. De facto, a aplicação é capaz de transferir imagens entre 2 computadores através de uma porta-série, sendo o envio bem sucedido mesmo em situações de quebra de ligação ou interferência, como ficou claro na demonstração do projeto.

Adicionalmente, a implementação foi feita em camadas, tendo sido observados os princípios de encapsulamento. Consideramos ainda que os conceitos e implicações práticas dos protocolos usados ficaram claros.


%%%%%%%%%%%%%%%%
% BIBLIOGRAPHY %
%%%%%%%%%%%%%%%%
%\bibliographystyle{IEEEtran}
%\bibliography{rabb,refs}

\begin{appendices}

%%%%%%%%%%%%%
% APENDICES %
%%%%%%%%%%%%%
\chapter{Código Fonte}

\input{./Appc.tex}
\input{./AppProtocolc.tex}
%%%%%%%%%%%%%%%%%%%%%%%%%%%
% APPPROTOCOL.c%
\section{AppProtocol.h}
\label{APPPROTOCOLH}
%%%%%%%%%%%%%%%%%%%%%%%%%%%

\begin{lstlisting}

#ifndef APPPROTOCOL
#define APPPROTOCOL

#define CD 0b00000000 // campo de controlo no caso de um pacote de dados
#define CS 0b00000001 // campo e controlo no caso de um pacote de start
#define CE 0b00000010 // campo e controlo no caso de um pacote de end
#define TSIZE 0b00000000 // type para pacote de controlo no caso de ser tamanho do ficheiro
#define TNAME 0b00000001 // type para pacote de controlo no caso de ser nome do ficheiro
#define START 1
#define END 2
#define L2 50
#define L1 100
#define MAX_CTRL_P 264 /* maximum size of control packet: 1 byte for C, 2 bytes for T and L, 4 bytes for size, 2 bytes for T and L, 255 bytes for name */
#define MAX_INFO_P 65540 /* maximum size of info packet: 1 byte for C, 1 byte for N, 2 bytes for L2 and L1, 255 * 256 + 255 bytes for info */
#define MAX_TRY 1

/**
@return 0(OK) if no errors/problems. -1 if failed to send everything, -2 failed on sending start
*/
int sendFile(unsigned char l2, unsigned char l1, int fd, unsigned char fileNameSize, const char *fileName, unsigned  int image_bytes_length, const char *image_bytes, unsigned  int* out_already_sent_bytes);

/**
@param out_imagename will get the name of the file to be used later. must be a 255 char array
@param out_imagebuffer will get the image bytes. Must be a dynamic array and be freed outside ths method
@return 0(OK) if no errors/problems. something else otherwise. -1 failed to receive everything, -1 failed to recive start
*/
int receiveFile(int fd, char* out_imagename, char** out_imagebuffer, unsigned int* out_image_buffer_length, unsigned  int* out_already_received_imgbytes);

#endif /*APPPROTOCOL*/

\end{lstlisting}

\input{./DataLinkProtocolc.tex}
\input{./DataLinkProtocolh.tex}
\input{./FileFuncsc.tex}
%%%%%%%%%%%%%%%%%%%%%%%%%%%
% FILEFUNCS.h%
\section{FileFuncs.h}
\label{FILEFUNCSH}
%%%%%%%%%%%%%%%%%%%%%%%%%%%

\begin{lstlisting}
#ifndef FILEFUNCS
#define FILEFUNCS

//copies file data to *dest_buf array. dest_buf must be dynamic and not initialized
//returns the length of the dest_buf
long getFileBytes(char* filename, char** dest_buf);

int save2File(char* data, int data_size, const char* filename);

#endif /*FILEFUNCS*/

\end{lstlisting}
%%%%%%%%%%%%%%%%%%%%%%%%%%%
% USER_INTERFACE.c%
\section{user\_interface.c}
\label{USERINTERFACEC}
%%%%%%%%%%%%%%%%%%%%%%%%%%%

\begin{lstlisting}
#include <stdlib.h>
#include <stdio.h>
#include <unistd.h>
#include <string.h>
#include "utilities.h"
#include "user_interface.h"
#include "FileFuncs.h"

//should have a start bigger than zero
int getIntPositiveRange(int start, int end) {
	int num;
	char get[50];
	int done = NO;

	while (!done) {
		//printf("(%d,%d)", start, end);

		gets(get);
		//fgets(get, 50, stdin); 
		///*clean buf*/char c; while ((c = getchar()) != '\n' && c != EOF);
		num = atoi(get);

		if (num != 0 && start <= num && num <= end)
			done = YES;
		else printf("Invalid input/range, please select again:\n==>");
	}

	return num;
}

char getAnswer(int numOfChoices)
{
	char get[20];
	do {
		gets(get);
		//fgets(get, 2, stdin); 
		///*clean buf*/char c; while ((c = getchar()) != '\n' && c != EOF);
		if (get[0] < 'a' || get[0] >= 'a' + numOfChoices || get[1] != 0)
			printf("\nOption not recognized, please select again:");

	} while (get[0] < 'a' || get[0] >= 'a' + numOfChoices || get[1] != 0);

	return get[0];
}

char main_menu(bool receiver)
{
	system("clear");

	printf("\nWELCOME TO THE APP! PLEASE SELECT ONE OF THE FOLLOWING OPERATONS");

	if (receiver)
		printf("\na) Establish Conection and receive picture");
	else
		printf("\na) Establish Conection and send picture");

	printf("\nb) Try to re-establish lost conection");
	printf("\nc) Change configurations");

	if (receiver != 0) printf("\nd) Choose path to save picture");
	else printf("\nd) Choose picture to send");

	printf("\ne) Show Ocurrences Log");

	printf("\nf) Quit\n\n==>");

	return getAnswer(6);
}


int select_config(void(*apply_options) (char, char, char, int))
{
	char boudOpt, reconectOpt, timetoutOpt;
	int packetSize;

	system("clear");

	printf("Select baudrate:");
	printf("\na)B0 - Hang Up   b)B50");
	printf("\nc)B75            d)B110");
	printf("\ne)B134           f)B150");
	printf("\ng)B200           h)B300");
	printf("\ni)B600           j)B1200");
	printf("\nk)B1800          l)B2400");
	printf("\nm)B4800          n)B9600");
	printf("\no)B19200         p)B38400 - Default");
	printf("\nq)B57600         r)B115200");
	printf("\ns)B230400        t)B460800\n=>");
	boudOpt = getAnswer(20);

	printf("\nSelect max Reconect Tries: \na)1 \nb)3 \nc)5 \nd)7\n");
	reconectOpt = getAnswer(4);

/*	printf("\nSelect timeout interval: \na)2 secs \nb)3 secs \nc)5 secs \nd)8 secs \n=>");
	timetoutOpt = getAnswer(4);
*/

	printf("\nInput packet size (number of file bytes per packet 1 - 65535):\n");
	packetSize = getIntPositiveRange(1, 65535);

	apply_options(boudOpt, reconectOpt, timetoutOpt, packetSize);

	return 0;
}




void show_prog_stats(unsigned long num_of_Is,
	unsigned long total_num_of_timeouts,
	unsigned long num_of_REJs, int appstatus)
{
	system("clear");

	printf("- - - Ocurrences log - - -");

	if (appstatus/*receiver*/)
		printf("\nNumber of Is received: %lu", num_of_Is);
	else /*transmitter*/ printf("\nNumber of I's sent (RR confirmed): %lu", num_of_Is);

	printf("\nTotal number of timeouts: %lu", total_num_of_timeouts);

	if (appstatus/*receiver*/)
		printf("\nTotak number of REJs sent: %lu", num_of_REJs);
	else /*transmitter*/printf("\nTotal number of REJs received: %lu", num_of_REJs);

	printf("\n- - - - - - - - - - - - - - -\nPress Enter key to return to pevious menu...\n");

	char get[20];

	gets(get);
	//fgets(get, 2, stdin); 
	///*clean buf*/char c; while ((c = getchar()) != '\n' && c != EOF);
}



unsigned int selectNload_image(char** image_buffer, char* out_image_name, unsigned char* out_image_name_length)
{
	system("clear");
	printf("Please input image file path (relative or not):\n>");

	char get_path[100];

	gets(get_path);
	//fgets(get_path, 100, stdin);
	///*clean buf*/char c; while ((c = getchar()) != '\n' && c != EOF);

	char* image_path;
	while (TRUE){
		image_path = realpath(get_path, NULL);
		if (image_path != NULL) break;
		printf("\nInvalid path/file. Please choose another.\n");

		gets(get_path);
		//fgets(get_path, 100, stdin); 
		///*clean buf*/char c; while ((c = getchar()) != '\n' && c != EOF);
	}

	long imageSize;
	if ((imageSize = getFileBytes(image_path, image_buffer)) < 0)
	{
		printf("\nFailed to load image.\n");
		return -1;
	}

	//get name from path
	image_path = strrchr(get_path, '/');
	if (image_path==NULL)
	{
		strcpy(out_image_name, get_path);
	}
	else {
		//(image_path - get_path + 1)
		strcpy(out_image_name, image_path+1);
	}

	*out_image_name_length = strlen(out_image_name);

	printf("\nImage sucessfully loaded.<%s , %ld bytes>\n", out_image_name, imageSize);
	sleep(2);
	return (unsigned int) imageSize;
}
\end{lstlisting}
%%%%%%%%%%%%%%%%%%%%%%%%%%%
% USER_INTERFACE.c%
\section{user\_interface.h}
\label{USERINTERFACEH}
%%%%%%%%%%%%%%%%%%%%%%%%%%%

\begin{lstlisting}
#ifndef USER_INTERFACE
#define USER_INTERFACE

char main_menu(bool receiver);

/*@brief display menu configuration
* @param apply_options must be a pointer to a func that receives a pointer to an array of 4 chars. The chosen options will be sent to ths method that should apply them.
*/
int select_config(void(*apply_options) (char, char, char, int));


//void* show_progress(void* args);


void show_prog_stats(unsigned long num_of_Is,
	unsigned long total_num_of_timeouts,
	unsigned long num_of_REJs, int appstatus);

unsigned int selectNload_image(char** image_buffer, char* out_image_name, unsigned char* out_image_name_length);


#endif /*USER_INTERFACE*/
\end{lstlisting}

\input{./utilitiesh.tex}

\chapter{Tipos de Tramas Usadas}
\label{tramas}

As tramas que utilizamos podem ser de três tipos:
\begin{itemize}
	\item Informação (I) - transportam dados;
	\item Supervisão (S) - são usadas para iniciar e terminar a emissão, assim como para responder a tramas do tipo I;
	\item Não Numeradas (U) - são usadas para responder a tramas de início e fim de emissão.
\end{itemize}

Todas as tramas são delimitadas pelas \emph{Flags} F - 01111110. Além disso, independentemente do tipo da trama, o cabeçalho é sempre o mesmo conjunto de 3 bytes:
\begin{enumerate}
	\item A - Campo de Endereço - 00000011 em comandos enviados pelo Emissor e respostas enviadas pelo Receptor, 00000001 na situação inversa;
	\item C - Campo de Controlo:
		\begin{itemize}
			\item tramas I - 00S00000, onde S é o bit que identifica a trama;
			\item tramas SET (\emph{set up}) - 00000111;
			\item tramas DISC (\emph{disconnect})- 00001011;
			\item tramas UA (\emph{unnumbered acknowledgment}) - 00000011;
			\item tramas RR (\emph{positive acknowledgment}) - 00R00001, onde R identifica a trama;
			\item tramas REJ (\emph{negative acknowledgment}) - 00R00101, onde R identifica a trama;
		\end{itemize}
	\item BCC1 (\emph{Block Check Character}) - Campo de Proteção do Cabeçalho - é obtido realizando a disjunção exclusiva bit a bit de A e C.
\end{enumerate} 

Se se tratar de uma trama I, após o cabeçalho temos:
\begin{itemize}
	\item $\text{D}_{\text{1}}, \text{D}_{\text{2}}, \ldots, \text{D}_{\text{N}}$ - bytes de dados;
	\item BCC2 - Campo de Proteção dos Dados - calculado de forma que exista um número par de 1s em cada bit dos dados, incluindo o BCC2.
\end{itemize}

As restantes tramas apenas têm o respetivo cabeçalho delimitado por flags.

%--------------------------------------------------------------------------------------------------------------------------------------------

\chapter{Diagrama de chamadas a funções (Fluxo)}
\label{flux}
\begin{sidewaysfigure}
\centering
\includegraphics[width=26cm]{_app_8c_a3c04138a5bfe5d72780bb7e82a18e627_cgraph.png}
\end{sidewaysfigure}

%---------------------------------------------------------------------------------------------------------------------------------------------

\chapter{Diagrama de Módulos}
\label{modulediagram}
\centering
\includegraphics[width=15cm]{rcom_module_diagram.png}

\chapter{Imagens da aplicação}
\label{imagens_app}

\begin{figure}[h]
	\centering
	\includegraphics[width=12cm]{log.png}
	\caption{Registo de ocorrências da aplicação}
	\label{logimg}
\end{figure}

\begin{figure}[h]
	\centering
	\includegraphics[width=12cm]{progress.png}
	\caption{Representação do progresso de envio/receção}
	\label{progressimg}
\end{figure}


\end{appendices}

\end{document}
